% !TEX root = Projektdokumentation.tex

% Es werden nur die Abkürzungen aufgelistet, die mit \ac definiert und auch benutzt wurden. 
%
% \acro{VERSIS}{Versicherungsinformationssystem\acroextra{ (Bestandsführungssystem)}}
% Ergibt in der Liste: VERSIS Versicherungsinformationssystem (Bestandsführungssystem)
% Im Text aber: \ac{VERSIS} -> Versicherungsinformationssystem (VERSIS)

% Hinweis: allgemein bekannte Abkürzungen wie z.B. bzw. u.a. müssen nicht ins Abkürzungsverzeichnis aufgenommen werden
% Hinweis: allgemein bekannte IT-Begriffe wie Datenbank oder Programmiersprache müssen nicht erläutert werden,
%          aber ggfs. Fachbegriffe aus der Domäne des Prüflings (z.B. Versicherung)

% Die Option (in den eckigen Klammern) enthält das längste Label oder
% einen Platzhalter der die Breite der linken Spalte bestimmt.
\begin{acronym}[WWWWW]
	\acro{Arktis}{Arktis IT solutions GmbH}
	\acro{GmbH}{Gesellschaft mit beschränkter Haftung}
	\acro{PMO}{Projekt Management Office}
	\acro{UG}{Untergeschoss}
	\acro{WLAN}{Wireless Local Area Network}
	\acro{VDSL}{Very High Speed Digital Subscriber Line}
	\acro{LAN}{Local Area Network}
	\acro{VLAN}{Virtual Local Area Network}
	\acro{DHCP}{Dynamic Host Configuration Protocol}
	\acro{MVC}[MVC]{Model View Controller}
	\acro{NatInfo}[\textsc{NatInfo}]{Natural Information System}
	\acro{Natural}[\textsc{Natural}]{Programmiersprache der Software AG}
	\acro{ORM}{Object-Relational Mapping}
	\acro{PHP}{Hypertext Preprocessor}
	\acro{SDK}{Software Development Kit}
	\acro{SQL}{Structured Query Language}
	\acro{SVN}{Subversion}
	\acro{XML}{Extensible Markup Language}
\end{acronym}
