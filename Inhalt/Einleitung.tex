% !TEX root = ../Projektdokumentation.tex
\section{Einleitung}
\label{sec:Einleitung}


\subsection{Projektumfeld} 
\label{sec:Projektumfeld}
Der Praktikumsbetrieb beschäftigt 20 Mitarbeiter, davon sollen 15 Personen das Ergebnis dieses Projektes nutzen können.

Falls ein anderes Projekt wichtig für dieses Abschlussprojekt ist, soll das deutlich abgegrenzt werden (\bzw schon im Antrag, siehe auch Projektabgrenzung~\ref{sec:Projektabgrenzung}).

Zum Umfeld gehören auch die Schnittstellen, s. \ref{sec:Projektschnittstellen}.

\subsection{Projektziel} 
\label{sec:Projektziel}
\begin{itemize}
	\item Worum geht es eigentlich?
	\item Was soll erreicht werden?
\end{itemize}


\subsection{Projektbegründung} 
\label{sec:Projektbegruendung}
\begin{itemize}
	\item Warum ist das Projekt sinnvoll (\zB Kosten- oder Zeitersparnis, weniger Fehler)?
	\item Was ist die Motivation hinter dem Projekt?
\end{itemize}


\subsection{Projektschnittstellen} 
\label{sec:Projektschnittstellen}
\begin{itemize}
	\item Mit welchen anderen Systemen interagiert die Anwendung (technische Schnittstellen)?
	\item Wer genehmigt das Projekt \bzw stellt Mittel zur Verfügung? 
	\item Wer sind die Benutzer der Anwendung?
	\item Wem muss das Ergebnis präsentiert werden?
\end{itemize}


\subsection{Projektabgrenzung} 
\label{sec:Projektabgrenzung}
\begin{itemize}
	\item Was ist explizit nicht Teil des Projekts (\insb bei Teilprojekten)?
\end{itemize}
