% !TEX root = ../Projektdokumentation.tex
\newpage
\section{Einleitung}
\label{sec:Einleitung}
Die folgende Projektdokumentation erläutert den Ablauf des IHK-Abschlussprojektes, das der Autor im Rahmen seiner Ausbildung zum Fachinformatiker für Systemintegration durchgeführt hat. Alle Einkaufspreise und Kalkulationen wurden abgeändert, da sie unter das Betriebsgeheimnis fallen. Aufgrund von Datenschutzbestimmungen müssen Personen und Organisationen, die im Zusammenhang mit diesem Projekt stehen, anonym bleiben und IP-Adressen und Gerätenamen aufgrund des Datenschutzes abgeändert werden. Daher werden Personen und Organisationen, die die Dienste der ARKTIS IT Solutions GmbH in Anspruch nehmen im Folgenden nur als Kunde bezeichnet. 

\subsection{Projektumfeld} 
\label{sec:Projektumfeld}
Der Ausbildungsbetrieb ist die ARKTIS IT Solutions GmbH (kurz ARKTIS GmbH) ist ein mittelständischer IT-Dienstleister mit Hauptsitz in Berlin. Die ARKTIS GmbH beschäftigt zur Zeit ca. 158 MitarbeiterInnen und bietet Technologielösungen mit den Schwerpunkten IT Security, IT-Infrastrukturmanagement, Integrierte Kommunikationslösungen, Intelligente Gebäudetechnik und Digitalisierung an. Das Projekt wurde in einer Schule durchegführt, welche       

Falls ein anderes Projekt wichtig für dieses Abschlussprojekt ist, soll das deutlich abgegrenzt werden (\bzw schon im Antrag, siehe auch Projektabgrenzung~\ref{sec:Projektabgrenzung}).

Zum Umfeld gehören auch die Schnittstellen, s. \ref{sec:Projektschnittstellen}.

\subsection{Projektziel} 
\label{sec:Projektziel}
\begin{itemize}
	\item Worum geht es eigentlich?
	\item Was soll erreicht werden?
\end{itemize}


\subsection{Projektbegründung} 
\label{sec:Projektbegruendung}
\begin{itemize}
	\item Warum ist das Projekt sinnvoll (\zB Kosten- oder Zeitersparnis, weniger Fehler)?
	\item Was ist die Motivation hinter dem Projekt?
\end{itemize}


\subsection{Projektschnittstellen} 
\label{sec:Projektschnittstellen}
\begin{itemize}
	\item Mit welchen anderen Systemen interagiert die Anwendung (technische Schnittstellen)?
	\item Wer genehmigt das Projekt \bzw stellt Mittel zur Verfügung? 
	\item Wer sind die Benutzer der Anwendung?
	\item Wem muss das Ergebnis präsentiert werden?
\end{itemize}


\subsection{Projektabgrenzung} 
\label{sec:Projektabgrenzung}
\begin{itemize}
	\item Was ist explizit nicht Teil des Projekts (\insb bei Teilprojekten)?
\end{itemize}
