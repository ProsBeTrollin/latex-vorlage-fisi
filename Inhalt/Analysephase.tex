% !TEX root = ../Projektdokumentation.tex
\section{Analysephase} 
\label{sec:Analysephase}

\subsection{Kundengespräch}
\label{sec:Kundengespräch}
Zu Beginn des Projekts wurde ein Kundengespräch per Telefonkonferenz durchgeführt. Anwesend war der IT-Beauftragte der Schule, ein weiterer Mitarbeiter der Arktis, und ich. In diesem Gespräch wurden die Anforderungen und der aktuelle Zustand des Schulnetzwerkes konkretisiert und dokumentiert. Im Anschluss überreichte der Kunde uns die Ergebnisse der Funkausleuchtung, welche vor dem Beginn des Projektes abgeschlossen wurde. Bei der Funkausleuchtung wurden die idealen Montagepunkte für die Access Points  und die benötigte Anzahl der Access Points ermittelt, indem an verschiedenen möglichen Installationspunkten die Signalstärke gemessen wurde, um sicherzustellen dass von jedem Teil des Schulgebäudes ein störungsfreier Netzwerkzugang möglich ist. Mit Hilfe der zusätzlichen Informationen konnte ein Ist - und Soll-Konzept im Anschluss an das Gespräch ausgearbeitet werden. 

\subsection{Ist-Analyse} 
\label{sec:IstAnalyse}
Der Kunde hat für seine festen Arbeitsplätze ein kabelgebundenes Netzwerk mit Zugang zum Internet über seinen Internet Provider. 
Die Glasfaserleitung vom Internet Provider kommt in einem Serverraum im 1. \ac{UG} an und geht dort auf einen \acs{VDSL}-Router, welcher als Gateway zum Internet agiert. Von der Local Area Network \acs{LAN}-Schnittstelle des Routers ist der Router per Kupferkabel mit einem Switch verbunden, welcher über ein Patchfeld mit den festen Arbeitsplätzen verbunden ist (siehe \Anhang{app:NetzwerkplanIST}). Die Endgeräte an den festen Arbeitsplätzen sind alle in einem Netz. Die Endgeräte bekommen ihre IP Adresse, Subnetzmaske und ihr Standardgateway vom Router per \acs{DHCP} dynamisch zugewiesen. Die Schule plant im Unterricht verstärkt mobile Endgeräte zu benutzen und benötigt für diese Geräte eine Anbindung ans Schulnetzwerk und Internet. Da der Kunde vor hat die Anzahl der Endgeräte im Schulnetzwerk wesentlich auszubauen, wird außerdem ein neues Netzwerkkonzept benötigt, welches besser für eine größere Anzahl an Endgeräten geeignet ist. 

\subsection{Soll-Analyse}
\label{sec:Soll-Analyse}
Der Kunde plant das Netzwerk zu modernisieren, ausfallsicherer zu gestalten und die Sicherheit des Datenverkehrs zu verbessern. Hierfür soll das Schulnetzwerk durch die Nutzung von \acs{VLAN}s in mehrere Teilnetze eingeteilt werden, welche den Datenverkehr sinnvoll voneinander trennen und dadurch für mehr Sicherheit und weniger Datenstaus sorgen. Die genutzten Switches sollen untereinander redundant verbunden sein um die Ausfallsicherheit des Netzwerkes zu verbessern. Die mobilen Endgeräte im Schulgebäude sollen über ein kabelloses Netzwerk Zugriff aufs Schulnetzwerk und Internet bekommen. Das Netzwerk soll ausreichend Kapazität haben um den mobilen Endgeräten eine schnelle und hochverfügbare Verbindung zum Schulnetz und Internet bereitzustellen. Durch die Funkausleuchtung wurde ermittelt, dass 71 Access Points für eine optimale Signalstärke im gesamten Gebäude benötigt werden.
Die eingesetzten Access Points sollen gleichzeitig im 2.4GHz -und 5GHz Frequenzband senden und werden zentral von einem WLAN Controller gesteuert.
Es sollen mehrere Netze ausgestrahlt werden (z.B. Gast, Schüler, Lehrer). 
Die Berechtigungen der Nutzer verschiedener Netzwerke sind durch ein auf dem WLAN-Controller integriertes Firewall-Regelwerk klar definiert und getrennt. 
Die Kommunikation innerhalb des WLANs soll nach aktuellen Sicherheitsstandards verschlüsselt werden. Das neue Netzwerklayout soll eine Sternförmige Topologie mit einem Cluster aus 2 Layer 3 Switches im Mittelpunkt haben. Die Layer 3 Switches übernehmen im Schulnetzwerk das Routing zwischen den Netzen. Es soll pro Stockwerk einen Serverschrank mit Layer 2 Switches geben, welche die Access Points im Gebäude über \ac{POE} mit Strom versorgen. Die Layer 2 Switches sollen über einen \ac{LAG} per Glasfaser mit dem Layer 3 Switchcluster redundant verbunden sein. Durch den LAG werden mehrere physikalische Verbindungen zwischen zwei Switches zu einer logischen Verbindung gebündelt. Der LAG sorgt für eine höhere Ausfallsicherheit. Hierzu wurde ein Netzwerkplan angelegt.

\begin{comment}
	\item Wie ist die bisherige Situation (\zB bestehende Programme, Wünsche der Mitarbeiter)?
	\item Was gilt es zu erstellen/verbessern?
\end{comment}
\subsection{Auswahl der Hardwarekomponenten nach Kundenanforderungen}
\label{app:Auswahl der Hardwarekomponenten nach Kundenanforderungen}
Die benötigte Hardware wurde durch Nutzung einer Entscheidungsmatrix ausgewählt und den Kundenanforderungen angepasst. Ein Beispiel anhand der Auswahl des Modells für den Layer 2 Switch steht in Tabelle 2. Bei den Switches gab es 2 Features, die vorhanden sein mussten(Stacking-Support und \ac{POE}Support). Diese zwei Aspekte haben die alternativen extrem eingeschränkt, sodass 

\subsection{Wirtschaftlichkeitsanalyse}
\label{sec:Wirtschaftlichkeitsanalyse}
Aufgrund des Betriebsgeheimnisses 

\subsection{Projektkosten}
\label{sec:Projektkosten}
Die Projektkosten für die Durchführung des Projektes setzen sich aus den Personal -und den Ressourcenkosten zusammen.
\subsubsection{Personalkosten}
Laut Arbeitsvertrag verdient ein Auszubildener bei der Arktis GmbH im dritten Lehrjahr pro Monat \eur{1000} brutto. 
Wenn man den errechneten Stundensatz von \eur{8,25} mal die Durchführungszeit von 40 Stunden nimmt, kommt man auf \eur{330,13} Gesamtpersonalkosten für meinen Arbeitsaufwand. Die komplette Rechnung der Personalkosten für meine Arbeit befindet sich im \Anhang{app:Personalkosten}. 
Der Mitarbeiter der mit mir die Installation vor Ort durchgeführt hat, wurde mit einem Stundensatz von \eur{25} die Stunde berechnet.  

\subsection{Amortisationsdauer}
\label{sec:Amortisationsdauer}
Eine genaue Armortisationsdauer ist für dieses Projekt schwer zu ermitteln, weil ich nur vermuten kann, wie die Prozesse des Kunden durch dieses Projekt verbessert werden könnten. Durch die Digitalisierung des Unterrichtsalltages können einige Materialkosten in Form von Papier und Toner eingespart werden. Darüber hinaus können durch ein Netzwerk mit besserer Ausfallsicherheit Fälle vermieden werden, wo die Produktivität unter Netzwerkausfällen leidet. Ein besser abgesichertes Netzwerk schützt außerdem vor wirtschaftlichen Schäden die bei Hackerangriffen entstehen können. 

\subsection{Qualitätsanforderungen}
\label{sec:Qualitaetsanforderungen}
\begin{itemize}
	\item Welche Qualitätsanforderungen werden an die Anwendung gestellt (\zB hinsichtlich Performance, Usability, Effizienz \etc 
\end{itemize}

\subsection{Schutzbedarf}
\label{sec:Schutzbedarf}
Die Analyse des Schutzbedarfes des kabellosen Netzwerkes habe ich nach Empfehlungen des \ac{BSI} (siehe ) durchgeführt. Das Schutzziel der Vertraulichkeit habe ich mit dem Schutzbedarf hoch eingestuft, weil im Schulnetzwerk auch vertrauliche Informationen transportiert werden, welche vor Zugriff unbefugter Personen zu schützen sind. Die Integrität hat einen mittleren Schutzbedarf, da es zwar generell zu verhindern ist, dass Unbefugte in der Lage sind Daten zu manipulieren, aber fehlerhafte bzw. manipulierte Dateien eine geringe Auswirkung haben und leicht zu erkennen sind.


\begin{comment}
	\item Welcher Schutzbedarf  wird an die Anwendung gestellt (\zB hinsichtlich Sicherheit, Wichtigtkeit, ... \etc (siehe \citet{BSI-S-200-2}))?
\end{comment}

\subsection{Schutzmaßnahmen}
\label{sec:Schutzmaßnahmen}
Im Sinne der Vertraulichkeit des Netzwerkes, wurde ein Firewallregelwerk angelegt, welches pro Netzwerk klar definiert, auf welche anderen Netzwerke man zugreifen kann. Darüber hinaus wurde sich bei der Authentifizierungs -und Verschlüsselungsmethode für WPA2 mit \ac{PSK} als Anmeldemethode entschieden. Die Passwortwahl wurde dem Kunden überlassen, es wurde aber eine hohe Passwortkomplexität empfohlen, damit das Passwort von potentiellen Angreifern nicht zu leicht zu erraten ist. WPA3 wurde bei der Auswahl auch beachtet, da es einige Sicherheitslücken von WPA2  schließt. Da die Schule aber einige Endgeräte besitzt die veraltet sind und WPA3 teilweise nicht unterstützen würden, wurde es sich dagegen entschieden. Um die Verfügbarkeit zu schützen wurden die Switches untereinander redundant mit einem \ac{LAG} verbunden, wodurch die Ausfallsicherheit werden konnte.  
\begin{comment}
	\item Welche Schutznahmen werden unternommen um das System abzichern (\zB gegenüber fremden Zugriff, Sicherheitslücken, Updates, Ausfall des Systems \etc)?
\end{comment}
