% !TEX root = ../Projektdokumentation.tex
\section{Analysephase} 
\label{sec:Analysephase}
\subsection{Anforderungen aus der Ausschreibung}
\label{sec:Anforderungen aus der Ausschreibung}
Die Schule hat für den Auftrag eine Ausschreibung erstellt aus welchem sich einige Anforderungen ableiten ließen.
\subsection{Kundengespräch}
\label{sec:Kundengespräch}
Zu Beginn des Projekts wurde ein Kundengespräch per Telefonkonferenz durchgeführt. Anwesend war der IT-Beauftragte der Schule, ein weiterer Mitarbeiter der Arktis, und ich. In diesem Gespräch wurden die Anforderungen und der aktuelle Zustand des Schulnetzwerkes konkretisiert und dokumentiert. Im Anschluss überreichte der Kunde uns die Ergebnisse der Funkausleuchtung, welche vor dem Beginn des Projektes abgeschlossen wurde. Bei der Funkausleuchtung wurden die idealen Montagepunkte für die Access Points  und die benötigte Anzahl der Access Points ermittelt, indem an verschiedenen möglichen Installationspunkten die Signalstärke gemessen wurde, um sicherzustellen dass von jedem Teil des Schulgebäudes ein störungsfreier Netzwerkzugang möglich ist. Mit Hilfe der zusätzlichen Informationen konnte ein Ist - und Soll-Konzept im Anschluss an das Gespräch ausgearbeitet werden.

\subsection{Ist-Analyse} 
\label{sec:IstAnalyse}
Der Kunde hat für seine festen Arbeitsplätze ein kabelgebundenes Netzwerk mit Zugang zum Internet über seinen Internet Provider. 
Die Glasfaserleitung vom Internet Provider kommt in einem Serverraum im 1. \ac{UG} an und geht dort auf einen \ac{VDSL}-Router, welcher als Gateway zum Internet agiert. Von der \ac{LAN}-Schnittstelle des Routers ist der Router per Kupferkabel mit einem Switch verbunden, welcher über ein Patchfeld mit den festen Arbeitsplätzen verbunden ist (siehe NetzwerkplanIST). Die Endgeräte an den festen Arbeitsplätzen sind alle in einem Netz. Die Endgeräte bekommen ihre IP Adresse, Subnetzmaske und ihr Standardgateway vom Router per \ac{DHCP} dynamisch zugewiesen. Die Schule plant im Unterricht verstärkt mobile Endgeräte zu benutzen und benötigt für diese Geräte eine Anbindung ans Schulnetzwerk und Internet. Da der Kunde vor hat die Anzahl der Endgeräte im Schulnetzwerk wesentlich auszubauen, wird außerdem ein neues Netzwerkkonzept benötigt, welches besser für eine größere Anzahl an Endgeräten geeignet ist. 

\subsection{Soll-Analyse}
\label{sec:Soll-Analyse}
Der Kunde plant das Netzwerk zu modernisieren, ausfallsicherer zu gestalten und die Sicherheit des Datenverkehrs zu verbessern. Hierfür soll das Schulnetzwerk durch die Nutzung von \ac{VLAN}s in mehrere Teilnetze eingeteilt werden, welche den Datenverkehr sinnvoll voneinander trennen und dadurch für mehr Sicherheit und weniger Datenstaus sorgen. Die genutzten Switches sollen untereinander redundant verbunden sein um die Ausfallsicherheit des Netzwerkes zu verbessern. Das Netzwerk soll ausreichend Kapazität haben um den mobilen Endgeräten eine schnelle und hochverfügbare Verbindung zum Schulnetz und Internet bereitzustellen. 
Die eingesetzten Access Points sollen gleichzeitig im 2.4GHz -und 5GHz Frequenzband senden und empfangen um die Kompatibilität . 
Es sollen mehrere Netze ausgestrahlt werden (z.B. Gast, Schüler, Lehrer). 
Abhängig davon, über welches Netz man sich im WLAN anmeldet, soll man verschiedene Berechtigungen im Schulnetz bekommen. 
Die Kommunikation innerhalb des WLANs soll nach aktuellen Sicherheitsstandards verschlüsselt werden.

\begin{itemize}
	\item Wie ist die bisherige Situation (\zB bestehende Programme, Wünsche der Mitarbeiter)?
	\item Was gilt es zu erstellen/verbessern?
\end{itemize}


\subsection{Wirtschaftlichkeitsanalyse}
\label{sec:Wirtschaftlichkeitsanalyse}
\begin{itemize}
	\item Lohnt sich das Projekt für das Unternehmen?
\end{itemize}

\subsection{Projektkosten}
\label{sec:Projektkosten}
Die Projektkosten für die Durchführung des Projektes setzen sich aus den Personal -und den Ressourcenkosten zusammen.
\subsubsection{Personalkosten}
Laut Arbeitsvertrag verdient ein Auszubildener bei der Arktis GmbH im dritten Lehrjahr pro Monat \eur{1000} brutto. 
Wenn man den errechneten Stundensatz von \eur{8,25} mal die Durchführungszeit von 40 Stunden nimmt, kommt man auf \eur{330,13} Gesamtpersonalkosten für meinen Arbeitsaufwand. Die komplette Rechnung der Personalkosten für meine Arbeit befindet sich im \Anhang{app:Personalkosten}. 
Der Mitarbeiter der mit mir die Installation vor Ort durchgeführt hat, wurde mit einem Stundensatz von \eur{25} die Stunde berechnet.  

\subsubsection{Amortisationsdauer}
\label{sec:Amortisationsdauer}
\begin{itemize}
	\item Welche monetären Vorteile bietet das Projekt (\zB Einsparung von Lizenzkosten, Arbeitszeitersparnis, bessere Usability, Korrektheit)?
	\item Wann hat sich das Projekt amortisiert?
\end{itemize}

\paragraph{Beispielrechnung (verkürzt)}
Bei einer Zeiteinsparung von 10 Minuten am Tag für jeden der 25 Anwender und 220 Arbeitstagen im Jahr ergibt sich eine gesamte Zeiteinsparung von 
\begin{eqnarray}
25 \cdot 220 \mbox{ Tage/Jahr} \cdot 10 \mbox{ min/Tag} = 55000 \mbox{ min/Jahr} \approx 917 \mbox{ h/Jahr} 
\end{eqnarray}

Dadurch ergibt sich eine jährliche Einsparung von 
\begin{eqnarray}
917 \mbox{h} \cdot \eur{(25 + 15)}{\mbox{/h}} = \eur{36680}
\end{eqnarray}

Die Amortisationszeit beträgt also $\frac{\eur{2739,20}}{\eur{36680}\mbox{/Jahr}} \approx 0,07 \mbox{ Jahre} \approx 4 \mbox{ Wochen}$.


\subsection{Anwendungsfälle}
\label{sec:Anwendungsfaelle}
\begin{itemize}
	\item Welche Anwendungsfälle soll das Projekt abdecken?
	\item Einer oder mehrere interessante (!) Anwendungsfälle könnten exemplarisch durch ein Aktivitätsdiagramm oder eine \ac{EPK} detailliert beschrieben werden. 
\end{itemize}

\paragraph{Beispiel}
Ein Beispiel für ein Use Case-Diagramm findet sich im \Anhang{app:UseCase}.


\subsection{Qualitätsanforderungen}
\label{sec:Qualitaetsanforderungen}
\begin{itemize}
	\item Welche Qualitätsanforderungen werden an die Anwendung gestellt (\zB hinsichtlich Performance, Usability, Effizienz \etc (siehe \citet{ISO9126}))?
\end{itemize}

\subsection{Schutzbedarf}
\label{sec:Schutzbedarf}
Die Analyse des Schutzbedarfes des kabellosen Netzwerkes habe ich nach Empfehlungen des \ac{BSI} durchgeführt. Das Schutzziel der Vertraulichkeit habe ich mit dem Schutzbedarf hoch eingestuft, weil im Schulnetzwerk auch vertrauliche Informationen transportiert werden, welche vor Zugriff unbefugter Personen zu schützen sind. Die Integrität hat einen mittleren Schutzbedarf, da es zwar generell zu verhindern ist , dass Unbefugte in der Lage sind Daten zu manipulieren, aber fehlerhafte bzw. manipulierte Dateien eine geringe Auswirkung haben und leicht zu erkennen sind.


\begin{itemize}
	\item Welcher Schutzbedarf  wird an die Anwendung gestellt (\zB hinsichtlich Sicherheit, Wichtigtkeit, ... \etc (siehe \citet{BSI-S-200-2}))?
\end{itemize}

\subsection{Schutzmaßnahmen}
\label{sec:Schutzmaßnahmen}
\begin{itemize}
	\item Welche Schutznahmen werden unternommen um das System abzichern (\zB gegenüber fremden Zugriff, Sicherheitslücken, Updates, Ausfall des Systems \etc)?
\end{itemize}

%%\subsection{Lastenheft/Fachkonzept}
%%\label{sec:Lastenheft}
%%\begin{itemize}
	%\item Auszüge aus dem Lastenheft/Fachkonzept, wenn es im Rahmen des Projekts erstellt wurde.
	%\item Mögliche Inhalte: Funktionen des Programms (Muss/Soll/Wunsch), User Stories, Benutzerrollen
%%\end{itemize}

%\paragraph{Beispiel}
%Ein Beispiel für ein Lastenheft findet sich im \Anhang{app:Lastenheft}. 
