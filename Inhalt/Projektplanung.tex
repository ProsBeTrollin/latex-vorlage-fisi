% !TEX root = ../Projektdokumentation.tex
\section{Projektplanung} 
\label{sec:Projektplanung}


\subsection{Projektphasen}
\label{sec:Projektphasen}

Das Projekt wurde in 5 Arbeitstagen im Zeitraum vom 14.04.2023 bis zum 31.05.2023 durchgeführt. Das Projekt ist in die vier Phasen \glqq Planungsphase\grqq{},\glqq Implementierungsphase\grqq{},\glqq Deploymentphase\grqq{} und \glqq Projektabschluss\grqq{} eingeteilt, wie in Tabelle 1 zu sehen ist.
\begin{comment}

	\item Verfeinerung der Zeitplanung, die bereits im Projektantrag vorgestellt wurde.
\paragraph{Beispiel}
Tabelle~\ref{tab:Zeitplanung} zeigt ein Beispiel für eine grobe Zeitplanung.
\end{comment}
\tabelle{Zeitplanung}{tab:Zeitplanung}{ZeitplanungKurz}\\
Eine detailliertere Zeitplanung findet sich im \Anhang{app:Zeitplanung}.


\subsection{Abweichungen vom Projektantrag}
\label{sec:AbweichungenProjektantrag}
Der detailliertere Projektplan hat sich seit dem Projektantrag etwas verändert, da sich manche Anforderungen geändert haben. Der Kunde hatte angemerkt, dass zusätzlich zu der im Lieferumfang enthaltenen \glqq Limited Lifetime Warranty\grqq{}, welche Geräte bis zum offiziellen \glqq End of Life\grqq{} absichert(siehe \citet{LLW}), auch noch eine Garantieerweiterung, die Ersatzgeräte zum nächsten Werktag beinhaltet, bestellt werden sollte. Damit diese Garantieerweiterung aktiv wird, mussten die Netzwerkgeräte zusätzlich inventarisiert und bei LANCOM auf der Webseite aktiviert werden.
\begin{comment}
	\item Sollte es Abweichungen zum Projektantrag geben (\zB Zeitplanung, Inhalt des Projekts, neue Anforderungen), müssen diese explizit aufgeführt und begründet werden.
\end{comment}


\subsection{Ressourcenplanung}
\label{sec:Ressourcenplanung}

\begin{itemize}
	\item Detaillierte Planung der benötigten Ressourcen (Hard-/Software, Räumlichkeiten \usw).
	\item \Ggfs sind auch personelle Ressourcen einzuplanen (\zB unterstützende Mitarbeiter).
	\item Hinweis: Häufig werden hier Ressourcen vergessen, die als selbstverständlich angesehen werden (\zB PC, Büro). 
\end{itemize}


\subsection{Entwicklungsprozess}
\label{sec:Entwicklungsprozess}
\begin{itemize}
	\item Welcher Entwicklungsprozess wird bei der Bearbeitung des Projekts verfolgt (\zB Wasserfall, agiler Prozess)?
\end{itemize}
