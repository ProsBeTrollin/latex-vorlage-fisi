% !TEX root = ../Projektdokumentation.tex
\section{Fazit} 
\label{sec:Fazit}

\subsection{Soll-/Ist-Vergleich}
\label{sec:SollIstVergleich}

\begin{itemize}
	\item Projektziel erreicht?
	\item Ist der Auftraggeber mit dem Projektergebnis zufrieden und wenn nein, warum nicht?
	\item Wurde die Projektplanung (Zeit, Kosten, Personal, Sachmittel) eingehalten oder haben sich Abweichungen ergeben und wenn ja, warum?
	\item Hinweis: Die Projektplanung muss nicht strikt eingehalten werden. Vielmehr sind Abweichungen sogar als normal anzusehen. Sie müssen nur vernünftig begründet werden (\zB durch Änderungen an den Anforderungen, unter-/überschätzter Aufwand).
\end{itemize}

\paragraph{Beispiel (verkürzt)}
Wie in Tabelle~\ref{tab:Vergleich} zu erkennen ist, konnte die Zeitplanung bis auf wenige Ausnahmen eingehalten werden.
\tabelle{Soll-/Ist-Vergleich}{tab:Vergleich}{Zeitnachher.tex}


\subsection{Reflexion}
\label{sec:Reflexion}
Dies war das erste Projekt, welches ich eigenständig geplant und durchgeführt habe. Ich konnte vieles über das Projektmanagement und die damit zusammenhängenden Prozesse lernen. Meine Kommunikationsfähigkeiten konnte ich durch den Austausch mit dem Kunden und anderen Mitarbeitern verbessern. Durch 
\begin{comment}
	\item Was hat der Prüfling bei der Durchführung des Projekts gelernt (\zB Zeitplanung, Vorteile der eingesetzten Frameworks, Änderungen der Anforderungen)?
\end{comment}


\subsection{Ausblick}
\label{sec:Ausblick}
Der Kunde plant in naher Zukunft weitere Erweiterungen des Schulnetzwerkes. Auf den VM Servern sollen mehr Dienste bereitgestellt werden. Die IT-Abteilung arbeitet zur Zeit beispielsweise an der Einrichtung eines Moodle Servers. Das Captive Portal des GAST-WLAN soll visuell überarbeitet werden. Im Rahmen des Hauptprojektes soll der Standort in Zukunft zusätzlich noch an die LANCOM Cloud angebunden werden. 
\begin{comment}
	\item Wie wird sich das Projekt in Zukunft weiterentwickeln (\zB geplante Erweiterungen)?
\end{comment}
