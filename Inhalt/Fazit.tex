% !TEX root = ../Projektdokumentation.tex
\section{Fazit} 
\label{sec:Fazit}

\subsection{Soll-/Ist-Vergleich}
\label{sec:SollIstVergleich}
Das Projektziel konnte erreicht werden und der Kunde wirkte im Feedback stets zufrieden. Die Zeitplanung der einzelnen Projektphasen konnte eingehalten werden. 

\subsection{Reflexion}
\label{sec:Reflexion}
Dies war das erste Projekt, welches ich eigenständig geplant und durchgeführt habe. Ich konnte vieles über das Projektmanagement und die damit zusammenhängenden Prozesse lernen. Meine Kommunikationsfähigkeiten konnte ich durch den Austausch mit dem Kunden und anderen Mitarbeitern verbessern. Es gab sehr viele kleine Änderungen in den Anforderungen an die Konfiguration und Termine mussten mehrfach verschoben werden. Das hat es für mich extrem schwer gemacht, die Übersicht zu behalten aber ich konnte dadurch auch wertvolle Erfahrung sammeln.
\begin{comment}
	\item Was hat der Prüfling bei der Durchführung des Projekts gelernt (\zB Zeitplanung, Vorteile der eingesetzten Frameworks, Änderungen der Anforderungen)?
\end{comment}
\subsection{Optimierungsmöglichkeiten}
\label{sec:Optimierungsmöglichkeiten}
Es fiel mir schwer mich an vorgegebene Zeitrahmen zu halten, weil mein Zeitmanagement noch nicht routiniert genug war war. Generell fehlte mir die Erfahrung um einzuschätzen, wie lange bestimmte Aktivitäten dauern und dadurch habe ich den Zeitumfang mancher Aufgaben etwas unterschätzt. Meine Kommunikation mit dem Kunden war zum Anfang des Projektes nicht präzise genug, wodurch Missverständnisse auftreten könnten.


\subsection{Ausblick}
\label{sec:Ausblick}
Der Kunde plant in naher Zukunft weitere Erweiterungen des Schulnetzwerkes. Auf den VM Servern sollen mehr Dienste bereitgestellt werden. Die IT-Abteilung arbeitet zur Zeit beispielsweise an der Einrichtung eines Moodle Servers. Das Captive Portal des GAST-WLAN soll visuell überarbeitet werden. Im Rahmen des Hauptprojektes soll der Standort in Zukunft zusätzlich noch an die LANCOM Cloud angebunden werden. Es ist geplant, in Zukunft eine Administratorschulung zu entwickeln, welche den Mitarbeitern der IT-Abteilung eine EInführung in die Bedienung und Konfiguration der Installierten Netzwerkgeräte geben soll. 
\clearpage
\begin{comment}
	\item Wie wird sich das Projekt in Zukunft weiterentwickeln (\zB geplante Erweiterungen)?
\end{comment}
