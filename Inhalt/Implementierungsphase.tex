% !TEX root = ../Projektdokumentation.tex
\section{Implementierungsphase} 
\label{sec:Implementierungsphase}
\subsection{Bestellung}
\label{sec:Bestellung}
Bei der Bestellung der nötigen Arbeitsmaterialien wurde ich vom kaufmännischem Projektleiter unterstützt. Die \ac{Arktis} konnte dank dem LANCOM Partner Status Die Switches, Access Points und den WLAN-Controller von LANCOM direkt zu besseren Konditionen bestellen. Das restliche Material hat die \ac{Arktis} Standardlieferanten bestellt. Durch die langjährige Partnerschaft mit dem Händler konnte die Arktis auch hier zu besseren Konditionen einkaufen als es normalerweise möglich wäre.
\subsection{Inventarisierung und Lizenzaktivierung der Netzwerkgeräte}
\label{Inventarisierung und Lizenzaktivierung der Netzwerkgeräte}
Nachdem die bestellten Netzwerkgeräte bei der \ac{Arktis} im Lager angekommen sind, habe wurden mithilfe der anderen Auszubildenen der Abteilung IT-Infrastruktur alle Geräte inventarisiert und auf der LANCOM Webseite die Garantieerweiterung aktiviert. 

\subsection{Einrichten des Testaufbaus}
\label{sec:Einrichten des Testaufbaus}
Nach der Inventarisierung wurden die Access Points verschickt und die anderen Netzwerkkomponenten zum Testaufbau im Testlabor der Abteilung IT Infrastruktur aufgebaut. Die Switches wurden wie im Soll-Netzwerkplan über die dedizierten Stacking-Ports mit \ac{DAC} Kabeln zu einem logischen Switch verbunden. Die Switches und der WLAN Controller wurden mit Strom versorgt und über das Testnetz in der Laborumgebung erreichbar gemacht damit die Komponenten vorkonfiguriert werden können.


\subsection{Konfiguration der Layer 2 Switches}
\label{sec:Konfiguration der Layer 2 Switche}
Die Switches wurden über das Tool Lanconfig per WebUI konfiguriert. Lanconfig prüft über einen Broadcast, welche Lancom Geräte sich im gleichen Netzwerk befinden wie der PC der das Programm ausführt und zeigt die Geräte dann mit IP Adresse an. Für die Anbindung an die Layer 3 Switches wurde pro Stack zuerst der \ac{LAG} angelegt. Auf den  \ac{}






