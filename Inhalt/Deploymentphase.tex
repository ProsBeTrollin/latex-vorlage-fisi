% !TEX root = ../Projektdokumentation.tex
\section{Deploymentphase} 
\label{sec:Deploymentphase}
\subsection{Inbetriebnahme vor Ort}
\label{Inbetriebnahme vor Ort}
Der Termin zur Lieferung, Installation und Inbetriebnahme musste außerhalb der Schulzeiten passieren, weil die Arbeiten am Netzwerk sonst den Tagesbetrieb des Kunden zu stark stören würde. Die vorkonfigurierten Netzwerkkomponenten wurden von mir und einem Mitarbeiter  eingepackt und zum Kunden ausgeliefert. Da die Schule zuvor keine Verteilerräume hatte, waren die Serverschränke komplett neu und einige Portbeschriftungen auf den Patchfeldern fehlten, was die arbeiten verzögerte. Während noch die Informationen fehlten, wurden alle Netzwerkkomponenten in die richtigen Serverschränke eingebaut. Es wurden außerdem die \ac{DAC}-Kabel zum stacken der Switches gesteckt. Als der Kunde uns die nötigen Informationen mitgeteilt hat, wurden die Netzwerkkomponenten mit Strom versorgt. Hierbei wurde darauf geachtet, dass unterschiedliche Stromkreise verwendet wurden um die redundanten Netzteile (falls vorhanden) mit Strom zu versorgen. Die Serverräume hatten keine \ac{USV}. Nachdem die Switches hochgefahren sind, wurde überprüft ob der Stack sich richtig aufgebaut hat. Danach wurden alle Kabel entsprechend der Planung gesteckt und das Netzwerk wurde in Betrieb genommen.

\subsection{Qualitätskontrolle}
\label{app:Qualitätskontrolle}
Nach Inbetriebnahme des Netzwerks wurde geprüft ob alle Access Points erreichbar sind. Hierfür wurde erst auf dem Webinterface des WLAN Controllers bestätigt, dass alle Access Points erreichbar sind. im Anschluss wurde vom Core Cluster ein Ping-Test zu allen Switches und dem WLAN-Controller durchgeführt. Danach bin ich und der Kunde durchs Gebäude gelaufen, um die Status LEDs der Access Points zu prüfen. Die Signalstärke wurde mithilfe eines Testlaptops für das ganze Gebäude überprüft.